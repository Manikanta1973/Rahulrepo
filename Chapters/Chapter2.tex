% Chapter Template

\chapter{Background} % Main chapter title

\label{Chapter2} % Change X to a consecutive number; for referencing this chapter elsewhere, use \ref{ChapterX}

\lhead{Chapter 2. \emph{Background}} % Change X to a consecutive number; this is for the header on each page - perhaps a shortened title

%----------------------------------------------------------------------------------------
%	SECTION 1
%----------------------------------------------------------------------------------------

\section{Introduction}\label{Lit:Intro}
In this section I will layout the key ideas and technologies which have inspired and enabled the research presented later in this thesis.

Due to the broad scope of the research carried out, the background section is loosely presented so as to compare and contrast the state-of-the-art artificial neural network technologies with their biological analogues (where possible). As such, I survey a large body of machine learning, psychology and biology literature.

In doing this, I aim to demonstrate how ANNs have exceeded human performance in specific areas, whilst lagging far behind in many others. By bringing these divergent fields together, perhaps it is possible to find ways to improve ANNs whilst also further developing our understanding of what makes humans tick. 


\section{What are Artifical Neural Networks good at?}
Summary of sota literature demonstrating cool things ANNs can do.
\subsection{Classification}
\subsection{Generation}
\subsection{Reinforcement Learning}

\section{What are Artificial Neural Networks bad at?}
Explain how ANNs have specialised skill, not generalised skill - quote Rodney Brooks.
\subsection{Data Ineffciency}
look at oth ML methods which require less data but are generally less capable/require more engineering.


\section{How to get off the Symbol Grounding Merry-Go-Round} 
\subsection{What is symbol grounding?}
\subsection{How do humans do it?}
Barsalou

\subsection{How do machines do it?}

\section{Why brains are better}
\subsection{Embodiment}
\subsubsection{Sensory Redundancy}
\subsubsection{Biological Filters}
Retina, shape of ear etc.
\subsection{Development}
\subsubsection{Biological Filters, again}
Superior Colliculous guides learning in the visual cortex by controlling attention.
\subsection{Pulling yourself up by the bootstraps}

\subsection{Machine Equivelancy}
\subsubsection{How do we simulate Embodiment for ANNs?}
\subsubsection{How do we simulate Development for ANNs?}

\section{Where do we go from here?}
\subsection{Robot bodies}
\subsection{multimodality}
\subsection{transfer learning}


